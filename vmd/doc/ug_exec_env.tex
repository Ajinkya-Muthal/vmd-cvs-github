%%%%%%%%%%%%%%%%%%%%%%%%%%%%%%%%%%%%%%%%%%%%%%%%%%%%%%%%%%%%%%%%%%%%%%%%%%%%
% RCS INFORMATION:
%
%       $RCSfile: ug_exec_env.tex,v $
%       $Author: johns $        $Locker:  $                $State: Exp $
%       $Revision: 1.55 $      $Date: 2012/01/10 19:32:56 $
%
%%%%%%%%%%%%%%%%%%%%%%%%%%%%%%%%%%%%%%%%%%%%%%%%%%%%%%%%%%%%%%%%%%%%%%%%%%%%
% DESCRIPTION:
%   execution environment - command line, startup files, env. variables, etc.
%
%%%%%%%%%%%%%%%%%%%%%%%%%%%%%%%%%%%%%%%%%%%%%%%%%%%%%%%%%%%%%%%%%%%%%%%%%%%%
\chapter{Customizing \VMD\ Sessions}
\label{ug:topic:customizing}
\index{VMD!customizing}
There are a number of ways to change the behavior of \VMD\ from the
default settings, both in how the program starts up and in how the
program behaves during a session.  This Chapter describes the data files,
command-line options, and environment variables which are used to
customize a \VMD\ session.

These files control the initial appearance and behavior of \VMD\ at
the start, and may be customized to suit each user's particular tastes.
Default versions of these files are placed in the \VMD\ installation
directory (On Unix this is usually {\tt /usr/local/lib/vmd}, on Windows
this defaults to {\tt \verb+C:\Program Files\University of Illinois\VMD+} ).  
Each user may specify
their own versions of some of these files, but unless this is done the
commands and values in the default files are used.  In this way, an
administrator may customize the default behavior of \VMD\ for all
users, while giving each user the option to change the default
behavior however they choose.

Several configurable parameters may also be set in a number of ways,
including use of command-line options or environment variables.  The
order of precedence of these methods is as follows (highest precedence
to lowest):

\begin{enumerate}
  \item Command-line options.
  \item Environment variable settings.
  \item Built-in defaults, as specified by compilation configurable
parameters.  These are used only if no other values are specified by
the other methods mentioned in this list.  The Installation Guide
describes how to change these default values when compiling \VMD.
\end{enumerate}

\section{VMD Command-Line Options}
\index{VMD!command line options}
\index{Command line options}

When started, the following command-line options may be given to \VMD. Note
that if a command-line option does not start with a dash (-), and is not
part of another option, it is assumed to be a filename, and its extension is
matched to the extension registered by the proper plugin. Thus, the Unix
command
\begin{verbatim}
        vmd molecule.pdb 
\end{verbatim}
will start \VMD\ and load a molecule from the file {\tt molecule.pdb}, while
the command
\begin{verbatim}
        vmd molecule.psf molecule.dcd
\end{verbatim}
will load the corresponding structure and coordinate files into the same
molecule.
On the Windows platform, one must preface the \VMD\ invocation
with the Windows {\tt start} command 
\begin{verbatim}
        start vmd molecule.pdb 
\end{verbatim}

\begin{itemize}
  \item {\tt -h | --help} : 
     Print a summary a command-line options to the console. 
  \item {\tt -e filename} : 
     After initialization, execute the text commands in filename, and
     then resume normal operation. \index{play!command}
  \item {\tt -python} :
     After initialization, switch to the Python interpreter before executing
     commands in the file specified by {\tt -e} (if any), and leave the text 
     interpreter in Python mode. \index{python!startup}
  \item {\tt filename} :
     Load the specified file at startup.  The file type will be determined
     from the filename extension; if there is no filename extension, and
     the filename contains 4 letters, it is assumed to be a PDB accession
     code and will be loaded accordingly; otherwise the format is assumed
     to be PDB.
  \item {\tt -<filetype> filename} :
     Load the specified file using the given filetype.
  \item {\tt -f} :
    Load all subsequent files into the same molecule.  This is the default.
    A new molecule is created for each invocation of {\tt -f}; thus,
    {\tt vmd -f 1.pdb 2.pdb -f 3.pdb} loads {\tt 1.pdb and 2.pdb} into the
    same molecule and {\tt 3.pdb} into a different molecule.
  \item {\tt -m} :
    Load all subsequent files into separate molecules.  The {\tt -f} and
    {\tt -m} options may be specified multiple times on the command line
    in order to load multiple molecule containing one or more files.
  \item {\tt -dispdev < win | text | cave | caveforms | none >} : 
\index{display!device}
   Specify the type of graphical display to use. The possible display devices
   include: 
     \begin{itemize}
       \item {\tt win}: a standard graphics display window.
       \item {\tt text}: do not provide any graphics display window. 
       \item {\tt cave}: use the CAVE virtual environment for display,
                         windows are disabled.
       \item {\tt caveforms}: use the CAVE virtual environment for display
          and with windows enabled.  This is useful with {\tt -display machine:0}
          for remote display of the windows when the CAVE uses the local screen.
       \item {\tt none}: same as text. 
     \end{itemize}
     It is possible to use \VMD\ as a filter to convert coordinate files
     into rendered images, by using the {\tt-dispdev text} and {\tt -e} options.
  \item {\tt -dist z} : 
     Specify the distance to the \VMD\ image plane.
  \item {\tt -height y} : 
     Specify the height of the \VMD\ image plane.
  \item {\tt -pos x y} : 
     Specify the position for the graphics display window. The position
     (x,y) is the number of pixels from the lower-left corner of the
      display to the lower-left corner of the graphics window. 
  \item {\tt -size x y} :
     Specify the size for the graphics display window, in pixels. 
  \item {\tt -nt} :
     Do not display the \VMD\ title at startup. 
  \item {\tt -eofexit} :
\index{batch mode}
     Make VMD exit when EOF on stdin is reached; for example, when a script
       is redirected to VMD.  The combination 
       {\tt vmd -dispdev text -eofexit < input.tcl > output.log} 
       is useful for batch mode scripting.
  \item {\tt -startup filename} :
\index{startup files}
\index{files!startup}
   Use filename as the \VMD\ startup command script, instead of the
default {\bf .vmdrc} or {\bf vmd.rc} file.
\item {\tt -args} :
  Pass subsequent command line arguments to the text interpreter.  The Tcl
  interpreter will store these arguments in the list variable {\tt argv}.
  By default, no arguments are stored in this variable.
  \item {\tt -debug [level} :
   Turn on output of debugging messages, and optionally set the current debug
   level (1=few messages ... 5=many verbose messages). Note this is only useful
   if \VMD\ has been compiled with debugging option included. 
\end{itemize}

\section{Environment Variables}
\label{ug:exec_env:variables}
\index{environment variables}

Several environment variables are used by \VMD\ to determine the
location of certain files and directories. These variables are
accessible to text interface through array {\tt env}.
\index{variables!env}
These variables include:

\begin{itemize}
% XXX Unix-only feature
  \item {\tt DISPLAY} :
\index{environment variables!DISPLAY}
   {\bf (Unix-only)}
   The X-Windows display that \VMD\ should use for displaying the
   \VMD\ windows and menus, as well as the graphics window.  If this
   environment variable is not overridden by VMDGDISPLAY all 
   \VMD\ windows will be directed to this display.
    

  \item {\tt VMDDIR} :
\index{environment variables!VMDDIR}
   The directory which contains the \VMD\ data files (such as this help
   file) and architecture-specific executables. By default, this is
   /usr/local/lib/vmd on Unix systems, and 
   \verb+C:\Program Files\University of Illinois\VMD+ on Windows sytems.

  \item {\tt VMDTMPDIR} :
\index{environment variables!VMDTMPDIR}
   The directory which VMD should use for temporary data files. By
   default, this is /tmp, or /usr/tmp on Unix systems, and \verb+C:\+ 
   on Windows.

  \item {\tt VMDCUSTOMIZESTARTUP} :
   {\bf (Unix-only)}
\index{environment variables!VMDCUSTOMIZESTARTUP}
   The name of a C-shell script to source prior to running the actual
   VMD process.  This shell script can contain any commands necessary for
   performing machine-specific spaceball, graphics, and other customizations
   necessary to run VMD.  This can be anything from a simple script that 
   sets the right serial port for a Spaceball based on the hostname, or 
   it can be a complex script for turning on a projection system, logging
   demos, configuring multi-display stereo-framelock features, etc.

  \item {\tt VMDBABELBIN} :
\index{environment variables!VMDBABELBIN}
   The complete path and filename for the program babel, which is used
   by VMD to convert molecular structure/coordinates files into PDB
   files which \VMD\ can actually understand. If this is not set
   explicitly, the \VMD\ startup script will attempt to find babel in
   the current path. If Babel cannot be found or is not installed, \VMD\
   will not be able to read molecular file formats other than PDB,
   PSF, and binary DCD files.

  \item {\tt VMDFILECHOOSER} :
\index{environment variables!VMDFILECHOOSER}
   Specifies which file chooser to use for loading and saving files from the
   GUI.  At present, this should be either FLTK, which uses Fltk's platform-
   independent file chooser, or TK, which uses Tk's file chooser.  The Tk
   file chooser is the default and uses a native Windows interface on Windows 
   platforms. The Fltk file chooser looks the same on all platforms, supports
   tab completion but not drive letters, and is probably most appropriate for 
   Unix environments.  The file chooser can be overridden at any time by
   changing the environment variable (e.g., in Tcl, {\tt
   set env(VMDFILECHOOSER) FLTK}).  

  \item {\tt VMDFORCECPUCOUNT} :
\index{environment variables!VMDFORCECPUCOUNT}
   Specifies the maximum number of CPUs or CPU cores that VMD should
   use when running on a multiprocessor or multicore computer system.
   By default, VMD will use all of the processors on the host machine.
   This option can be used to prevent VMD from ``hogging'' CPUs or to
   make it abide by job submission policies required on large
   supercomputer systems, when running batch mode.

% XXX Unix-only feature
  \item {\tt VMDCAVEMEM} :
\index{environment variables!VMDCAVEMEM}
   {\bf (Unix-only)}
   This overrides the default size of the shared memory arena 
   which is allocated by VMD when the CAVE starts up.  The variable
   must be an integer number of megabytes.  Since this is the only 
   shared memory pool allocated, and it is done only once, you must
   choose a value sufficient to account for the largest scene you
   intend to render in VMD in that CAVE session.  The default value
   unless otherwise specified is 80 Megabytes.  Values of 200MB to 512MB
   are commonly needed for large molecular systems containing several
   hundred thousand atoms. 

% XXX Unix-only feature
  \item {\tt VMDFREEVRMEM} :
\index{environment variables!VMDFREEVRMEM}
   {\bf (Unix-only)}
   This overrides the default size of the shared memory arena 
   which is allocated by VMD when the FreeVR starts up.  The variable
   must be an integer number of megabytes.  Since this is the only 
   shared memory pool allocated, and it is done only once, you must
   choose a value sufficient to account for the largest scene you
   intend to render in VMD in that FreeVR session.  The default value
   unless otherwise specified is 80 Megabytes.  Values of 200MB to 512MB
   are commonly needed for large molecular systems containing several
   hundred thousand atoms. 

% XXX Unix-only, intended for clusters
  \item {\tt VMDFORCECONSOLETTY} :
\index{environment variables!VMDFORCECONSOLETTY}
   {\bf (Unix-only, intended only for clusters}
   This environment variable forces \VMD\ to treat the text console
   as an interactive terminal, despite what the operating system says.
   This is only useful for running an interactive VMD session on a 
   Clustermatic or Scyld Linux cluster node. 

% XXX Unix-only feature
  \item {\tt VMDGDISPLAY} :
\index{environment variables!VMDGDISPLAY}
   {\bf (Unix-only)}
   The name of an X-Windows display that VMD will use to display
   the graphics window.  This environment variable is only used
   on Unix systems.  Through the use of the the DISPLAY and
   VMDGDISPLAY envrironment variables, the VMD graphics window can
   be placed on a separate screen from the windows and menus.  This is
   particularly useful when giving 3-D demonstrations using a projector.
   The windows and menus can be kept on a different screen from the graphics
   so that they do not distract the audience.

  \item {\tt VMDGLSLVERBOSE} :
\index{environment variables!VMDGLSLVERBOSE}
   OpenGL Shading language compiler diagnostic errors only printed
   only when this environment variable is set.

  \item {\tt VMDHTMLVIEWER} :
\index{environment variables!VMDHTMLVIEWER}
   The name of a command that will run a web browser in the background 
   (Netscape, Mozilla, Firefox or whatever you prefer)
   that \VMD\ should use to display HTML documents (such as this help
   file). By default, on UNIX, this is mozilla. (usage examples in Tcl: 
   {\tt set env(VMDHTMLVIEWER) ``mozilla -remote openURL(\%s)"}, 
   {\tt set env(VMDHTMLVIEWER) ``mozilla \%s \&"})

  \item {\tt VMDIMAGEVIEWER} :
\index{environment variables!VMDIMAGEVIEWER}
   The name of the external program to use for displaying \VMD\
   snapshots (or other images), in various formats.

  \item {\tt VMDIMMERSADESKFLIP} :
\index{environment variables!VMDIMMERSADESKFLIP}
   Enable a special reversed/reflected stereo projection mode for 
   use with experimental displays based on LCD panels, phase plates, 
   and beam splitters.

  \item {\tt VMDVMDMACENABLEEEXTENSIONS} :
\index{environment variables!VMDVMDMACENABLEEEXTENSIONS}
   Enable performance-oriented OpenGL rendering extensions
   which are disabled by default.  These extensions have been 
   observed to trigger instability on some MacOS X systems.

  \item {\tt VMDMSECDELAYHACK} :
\index{environment variables!VMDMSECDELAYHACK}
   Add in a user-specified delay which causes \VMD\ to sleep for
   specified number of milliseconds each time it renders the molecular
   scene on the display.  This feature is meant as a workaround to 
   poor performing display drivers which make the windowing system
   unresponsive if VMD is allowed to run unrestricted at maximum drawing rate.

  \item {\tt VMDMSMSUSEFILE} :
\index{environment variables!VMDMSMSUSEFILE}
   Force \VMD\ to communicate with MSMS through the filesystem rather
   than with the socket-based network interface.  This option can be
   used when the socket interface isn't working properly for some reason.
   This is the default behavior when using \VMD\ on Windows.

  \item {\tt VMDNOCUDA} :
\index{environment variables!VMDNOCUDA}
   Force VMD not to use CUDA-based GPU acceleration, even if CUDA 
   support was compiled in, and CUDA-capable devices are detected at startup.
   This can be used in cases where a GPU or GPU drivers prove to be unreliable
   for computation, for benchmarking vs. CPU-only implementations, and
   to prevent VMD from using a GPU that's already oversubscribed by other
   processes running on the same machine.

% XXX Unix-only feature
  \item {\tt VMDDISABLESTEREO} :
\index{environment variables!VMDDISABLESTEREO}
   {\bf (Unix)}
   Prevents VMD from enabling stereoscopic display features.  This is
   normally only used as a workaround for buggy display drivers.

  \item {\tt VMDPREFERSTEREO} :
\index{environment variables!VMDPREFERSTEREO}
   {\bf (Unix, MacOS X)}
   On Unix systems using X11, this environment variable allows NVidia Quadro
   users to override the normal X11 visual search order, skipping multisample
   capable visuals in favor of stereo visuals.  VMD still attempts to get the  
   more complex visuals first, but if it comes down to a choice between stereo
   and multisample as mutually exclusive options, this variable provides 
   the ability to force the use of stereo if available.
   On MacOS X, this environment variable tells \VMD\ to create a
   a stereo-capable display window, even at the risk of terminating the 
   program if the request is denied.

  \item {\tt VMDSCRDIST} :
\index{environment variables!VMDSCRDIST}
   Distance to the \VMD\ image plane.

  \item {\tt VMDSCRHEIGHT} :
\index{environment variables!VMDSCRHEIGHT}
   Height of the \VMD\ image plane.

  \item {\tt VMDSCRPOS} :
\index{environment variables!VMDSCRPOS}
   Position of the \VMD\ graphics window (x,y).

  \item {\tt VMDSCRSIZE} :
\index{environment variables!VMDSCRSIZE}
   Size of the \VMD\ graphics window (x,y).

%%
%% Graphics rendering related controls
%%
  \item {\tt VMD\_EXCL\_GL\_EXTENSIONS} :
\index{environment variables!VMD\_EXCL\_GL\_EXTENSIONS}
  Disable the use of named OpenGL extensions according to their
  official OpenGL extension names.  This is intended to be used 
  only when one encounters severe stability problems caused by 
  buggy display drivers.


  \item {\tt VMDSHEARSTEREO} :
\index{environment variables!VMDSHEARSTEREO}
   Enable the use of an alternative perspective projection mode 
   which may result in improved stereoscopic display.  Uses the
   shear-matrix stereo formulation rather than eye rotation.


  \item {\tt VMDSIMPLEGRAPHICS} :
\index{environment variables!VMDSIMPLEGRAPHICS}
   Forces VMD to use absolutely minimalistic graphics features with no 
   use of OpenGL extensions.  Essentially, nothing but bread-and-butter 
   vertex arrays and immediate mode rendering will be used.  This mode
   is intended to be used only when one encounters severe stability 
   problems caused by buggy display drivers.

\item {\tt VMDWIREGL} :
\index{environment variables!VMDWIREGL}
   This environment variable disables several graphics features which
   are unsupported (or poorly supported) by WireGL and Chromium.
   This variable will be superceded with a more general
   implementation in a future release.

\end{itemize}

\section{Startup Files}
\index{startup files}
\index{files!startup}
\subsection{Core Script Files}

In the following, the value of {\tt \$VMDDIR} is the vmd installation
directory.  During the original installation this is the value of {\tt
INSTALLLIBDIR}.  It can also be found by looking at the first few
lines of the vmd startup script ({\tt head `which vmd`}) or by
starting \VMD\ and using the command {\tt set env(VMDDIR)}.
\index{variables!env}

As mentioned elsewhere, \VMD\ uses the Tcl interpreter.  \VMD\ 
read Tcl scripts at initialization, which are contained in \VMD\
distribution. The locations of the scripts is determined by the
{\tt TCL\_LIBRARY} environment variable, which
\index{environment variables!TCL\_LIBRARY}
is set in the vmd startup script to {\tt \$VMDDIR/scripts/tcl}. 
In addition, \VMD\ has its own directory of core Tcl routines.

The most important of these is {\tt \$VMDDIR/scripts/vmd/vmdinit.tcl}.  This
file sets up the basic Tcl initialization commands, defines some environment
variables, and adds the vmd script directory to the Tcl autoindex path.
Most of the other files are referenced through the {\tt auto\_path}.

There are a few non-Tcl scripts in this directory.  Currently these
are perl scripts used for the {\tt urlload} command and {\tt web
client} startup (see 
section \ref{ug:ui:text:mol} and section \ref{ug:topic:www:client}).

\subsection{User Script Files}

A user-written run-time command file, {\tt .vmdrc} on Unix,
{\tt vmd.rc} on Windows, can be used with a list
of initial \VMD\ text commands to process.  This file may be changed
to customize individual user's initial screen appearance and to set
the proper display characteristics for displaying in stereo.  If it
does not exist, default values are used.

\subsection{{\tt .vmdrc} and {\tt vmd.rc} Files}
\label{ug:section:vmdrc}
\index{{\tt .vmdrc}}
\index{{\tt vmd.rc}}
\index{startup files}
\index{files!startup}

After everything is initialized, \VMD\ reads the {\em startup} file
using the equivalent of the command {\tt play .vmdrc}.  
This file contains text commands for \VMD\ to execute just as if they had been
entered at the \VMD\ text console command prompt.  The file can
contain any number of commands, including blank lines and comment
lines (which begin with the {\tt \#} character).  If an error is
encountered while reading this file, the command in error is skipped
and processing of the file continues.

\VMD\ searches for this file in three locations on Unix;
{\tt ./.vmdrc}, {\tt \$HOME/.vmdrc} and {\tt \$VMDDIR/.vmdrc}.
On Windows, \VMD\ searches in
{\tt ./vmd.rc}, {\tt \$HOME/vmd.rc} and {\tt \$VMDDIR/vmd.rc}.
Only the first file found will be read in and processed.

%??? Idea is to refer to place for script -- go to new
%ug\_user\_scripts.tex file ???  
See chapter \ref{chapter:ug:text} 
for a description of the \VMD\ text
commands which may be put in this file.  Also,
section \ref{ug:ui:hotkeys} discusses how to put
commands into the {\tt .vmdrc} file to customize the behavior of the
hot keys.

Here is an example of a startup file:
\index{example scripts!Tcl!customized startup file example}
\begin{verbatim}
# add personalized keyboard shortcuts 
user add key E echo on
user add key e echo off
user add key g display reset
user add key A stage location bottom
user add key m mol list

# position the stage and axes
axes location lowerleft
stage location off

# position and turn on menus
menu main move 5 196
menu display move 386 90
menu graphics move 5 455
menu files move 5 496

menu main on

# start the scene a-rockin'
rock y by 1
\end{verbatim}

\section{Using \VMD\ as a WWW Client (for chemical/* documents)}
\label{ug:topic:www:client}

Mosaic, Netscape, and possibly other browsers can be configured to use VMD as a helper application for viewing some {\tt chemical/*} documents.

\subsection{MIME types}

\index{VMD!as MIME helper application}
When a web browser receives a document from a server it actually gets
two pieces of information: the header and the body.  The header
contains information about the message and body.  One of the most
important pieces of data, called the {\it MIME type} specifies what
the body of text describes.  For instance, a GIF image is given the
MIME type of {\tt image/gif}, a JPEG image is {\tt image/jpg}, and
postscript is {\tt application/postscript}.  A class of types, {\tt
chemical/*}, has been created for chemical models so the MIME type for
PDB files is {\tt chemical/pdb}, for XYZ is {\tt chemical/xyz}, etc.

\subsubsection{Helper Applications}

The web browser uses the MIME type to determine how to view the body
of the message.  Some of the documents are viewed by the browser
itself, like {\tt text/html} which describes HTML documents.  In other
cases, the browser has to start up another application.  From here on,
we'll describe how Mosaic and Netscape do this.  First, it saves the
incoming message body to a temporary file.  It then scans the global
and local {\it mailcap} files to determine which application is used
to view the given MIME type.  The application, which must take a file
name on the command line, is then executed.  When the application
exits, the temporary file is deleted.

\subsection{Setting up your {\tt .mailcap}}

\index{{\tt .mailcap}}
The Unix versions of \VMD\ have an extra {\tt -webhelper} 
command line flag which causes VMD not to be spawned in the 
background, so that it has time to read temporary files downloaded
by the web browser.  This command line flag is just slightly simpler
to use than the {\tt chemical2vmd} script, as it does not depend on having
Perl installed, so may be more appropriate for some cases.   

\index{chemical2vmd}
In the \VMD\ installation directory (\$VMDDIR/scripts/vmd/) 
there is a perl script called {\tt chemical2vmd} 
which will create a \VMD\ command file and execute VMD.  Since 
VMD does not block the calling process, Netscape and other web 
browsers cannot directly call VMD, as they do not know when to 
delete the temporary file containing the molecule or other data.
The {\tt chemical2vmd} script starts \VMD\ with the -e command line 
option which runs the saved VMD script or molecule file.

It is also possible to install the previous script in the global {\tt
.mailcap} file to make it accessible to everyone.  You will have to
consult the documentation for your web browser(s) to find out how.

\subsection{Example sites}

Some web sites that send {\tt chemical/pdb} types are the
Protein Data Bank at {\tt http://www.rcsb.org/} and ``Molecules R
US'' at {\tt http://www.nih.gov/htbin/pdb}.

