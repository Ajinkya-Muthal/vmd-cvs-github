
\subsection{Mouse Menu}
\label{ug:ui:window:mouse}
\index{window!mouse menu}

\begin{rawhtml}
<CENTER>
\end{rawhtml}
% \myfigure{ug_mouse}{Main Window Mouse Menu}{fig:ug:mouse}
\begin{rawhtml}
</CENTER>
\end{rawhtml}

The {\sf Mouse} menu indicates and controls the behavior of the mouse
when the mouse moves and clicks within the graphics window.  Mouse clicks
and drags can affect VMD in one of two ways.  It can change the {\em view}
of the scene, either by rotating, translating, or scaling.  It can also
{\em pick} objects in the scene, causing some further action to be taken.
These behaviors are all reflected in the state of the Mouse menu.  
            
Below, we describe the main parts of the Mouse menu.

\subsubsection{Mouse modes}
\index{mouse!mouse mode}

The top three menu items select whether the mouse will 
rotate, translate, or scale the scene
when the user clicks and drags with the left mouse button.  

\subsubsection{Pick modes}
\index{mouse!object menus}

These modes, located right below the mouse modes in the Mouse menu, 
control how the mouse affects objects in the scene (as opposed
to how the mouse changes the {\em view} of these objects).  Note that
any time you choose a new pick mode, the current mouse mode changes to "Rotate".  

\begin{itemize}
\item {\bf Center} changes how VMD rotates and scales the scene.
To get a feel for how this works, select "Center" from the Mouse menu, then
click on an atom in the scene.  If you now rotate the scene by clicking
and dragging with the left mouse button, the scene should rotate about the
picked atom.  If you change the view mode to "Scale" using the "View Mode"
pulldown menu, the scene will expand while keeping the picked atom in view.
The picked atom will remain the center atom until a new atom is selected as
"Center", the "Reset View" button is pressed, or a new molecule is loaded.

\item {\bf Query} prints information about the item (e.g. the atom name) on the
text console window.

\item {\bf Label} adds labels to atoms in the scene.  Labels
include atoms, bonds, angles, and dihedrals.  These labels require, 
respectively, one, two, three, and four atoms to be picked. For the latter 
three label types, the numerical value of the geometric label is displayed, 
along with a stippled line connecting the picked atoms.  The units for 
"Bonds" corresponds to whatever units the coordinate file is written in. 
"Angles" and "Dihedrals" are measure in degrees.  

Labels can then be manipulated through the {\sf Labels} window.
 
\index{mouse!move}
\index{mouse!move!atom}
\index{mouse!move!residue}
\index{mouse!move!fragment}
\index{mouse!move!molecule}
\index{mouse!move!highlighted rep}
\item{\bf Move} changes the actual coordinates of atoms in the scene.  Note
that this is different from simply changing the view.  Clicking on one of
the buttons in the Mode Mode menu selects what group of atoms to move.  "Atom"
moves only the selected atom.  "Residue" moves all atoms in the same residue
(e.g., amino acid or nucleotide) as the selected atom.  "Fragment" moves all 
atoms connected by a bond to the picked atom.  "Molecule" moves
every atom in the molecular structure.  "Highlighted Rep" is the most 
flexible; it moves all atoms in the highlighted representation in the
browser window of the Graphics window.  

Atoms are moved by clicking and dragging with the left mouse button.  If
the {\sf shift} key is held while the mouse is moved, the affected atoms are
{\em rotated} about the selected atom.   Rotating atoms with the left button
rotates about the x or y axis of the screen; rotating with the middle or
right button rotates about an axis perpendicular to the screen.

Note that there is currently no way to undo Move operations, so the 
atom coordinates should first be saved to a file. 

\item{\bf Force} applies a force to selected atoms in a running simualtion. 
These forces will
be visible only if an IMD connection has been established.  Clicking and
dragging with the left mouse button will apply a force to the selected Atom,
Residue, or Fragment, as in Move Mode.  Clicking with the middle or right
button will cancel the force on the selected atoms.

\item{\bf Move Light} allows the lights to be positioned around the scene.  
Individual lights are turned on or off in the Display window.  Selecting
one of the lights in the Move Light menu rotates the selected light about
the origin. The Move Light Mode can also be cancelled by changing into any 
other pick mode or mouse mode.

\item {\bf Add/Remove Bonds} adds a bond between two clicked atoms if there is
not one present, and removes the bond otherwise. Both atoms must belong to the
same molecule.

\end{itemize}
 

 

