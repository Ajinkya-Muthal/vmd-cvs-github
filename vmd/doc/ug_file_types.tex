%%%%%%%%%%%%%%%%%%%%%%%%%%%%%%%%%%%%%%%%%%%%%%%%%%%%%%%%%%%%%%%%%%%%%%%%%%%%
% RCS INFORMATION:
%
%       $RCSfile: ug_file_types.tex,v $
%       $Author: johns $        $Locker:  $                $State: Exp $
%       $Revision: 1.29 $      $Date: 2007/12/20 02:09:55 $
%
%%%%%%%%%%%%%%%%%%%%%%%%%%%%%%%%%%%%%%%%%%%%%%%%%%%%%%%%%%%%%%%%%%%%%%%%%%%%
% DESCRIPTION:
%  file types \VMD\ can use
%
%%%%%%%%%%%%%%%%%%%%%%%%%%%%%%%%%%%%%%%%%%%%%%%%%%%%%%%%%%%%%%%%%%%%%%%%%%%%

\chapter{Loading A Molecule}
\label{ug:topic:filetypes}
\index{files!reading}
\index{file types!input} 
\index{topology files}
\index{trajectory!files}
\index{AMBER!files}
\index{CHARMM!files}
\index{NAMD!files}
\index{Gromacs!files}
\index{XPLOR!files}

The File menu is the primary means for loading molecules and other
data into \VMD.  The built-in file readers will load molecular structures 
from combinations of topology files, coordinate files, and trajectory files.  
Readers are also included for data such as potential maps, 
electron density maps, Grasp surface data, and arbitrary 3-D geometric 
data from Raster3D scene files.  
\VMD\ can load structures directly from Protein Data Bank
over the internet, provided that a network connection is present.  
Entering the four-character PDB accession code in the molecule file
broswer form will retrieve and load the structure over the network.

\section{Notes on common molecular file formats}
\VMD\ natively understands several popular molecular data file formats: 
PDB coordinate files, 
CHARMM, NAMD, and X-PLOR style PSF topology files, 
CHARMM, NAMD, and X-PLOR style DCD trajectory files, 
NAMD binary restart (coordinate) files,
AMBER structure (PARM) and trajectory (CRD) files  including
both the old format and the new formats used by AMBER 7.0, 
and Gromacs (e.g. GRO, G96, XTC, TRR) structure and trajectory files.  
These files may contain some redundant information and can be loaded 
in different combinations.

PDB files contains data about atoms, residues, segment names,
occupancy and beta factor, and one coordinate set.  
PSF and PARM files contain atoms, residues, segment names, residue
types, atomic mass and charge, and the bond connectivity.  
VMD supports four file formats used by Gromacs: GRO, G96, TRR and XTC.  
GRO and G96 files contain structure information including atoms, 
residue and segment data, and one coordinate set.  
CRD, DCD, TRR and XTC files contain only coordinate data (\timesteps).  
It should be noted that while PDB, GRO and G96 files were designed
to contain only one coordinate set, multiple files can be 
concatenated into one larger file to create a makeshift trajectory file
which can be loaded by \VMD.

When \VMD\ loads a file it requires information about atom names and
coordinates and tries to fill in the rest.  Since the PDB file
contains all this information, it does not need to be loaded with any
other data files.  However, the PDB file doesn't contain the atom
types, masses, and charges, so these are guessed or assigned default
values.  In particular, charges will be assigned a value of 0.0 if the
file does not contain explicit charge information.

A PSF file does not contain coordinate information so it must be
loaded along with a PDB or DCD file.  If a PDB and PSF are given there
is no missing data and \VMD\ makes no assumptions.  If a PSF and DCD
are given then only the chain identifier and occupancy and beta values
are missing so they are given a default value.
A PARM file is similar to a PSF in that it too contains no coordinate
information.  It must be loaded along with a CRD trajectory file.
If a PARM and CRD file are loaded together, then only the segname and
chain ID for the atoms are left blank.  
A CRD or DCD file can be specified along with the PDB, in which case
the PDB file will be read as normal, and then coordinate sets are read
from the DCD or CRD until the end of the file is reached.
Gromacs GRO and G96 files can be loaded on their own since they contain
the necessary atom data and coordinates.  They can also be loaded
along with TRR and XTC files to obtain trajectory data.
Additional coordinates from a PDB, CRD, or DCD file can be appended to
the current coordinate set using the {\sf Molecule File Browser} form.



\section{What happens when a file is loaded?}
\label{ug:topic:filetypes:loading}
When a coordinate file is loaded by itself (i.e. just a PDB, no PSF),
\VMD\ uses heuristics to replace missing values that would normally 
be provided by a structure file.  If necessary, \VMD\ does a 
distance-based bond search \index{bonds!determining} to determine
connectivity.  A bond is formed whenever two atoms are within 
($R_1 + R_2) * 0.6$ of each other, where $R_1$ and $R_2$ 
are the respective radii of candidate atoms.
If both structure and coordinate files are loaded, 
no approximations or guesses are made.  

After the molecule is read in, new names are added to the
\hyperref{coloring categories}{coloring categories [\S }{]}
         {ug:topic:coloring:categories},
and assigned colors.
Next, bond connectivity is established and the molecule is
analyzed to identify its components, i.e., to determine which
residues are protein, nucleic acids, and waters, etc.
A search is then made to connect these into larger fragments of the same type,
and summary information is printed to the screen.  
An example output for BPTI is:
\begin{verbatim}
        Info 1) Analyzing structure ...
        Info 1)    Atoms: 898   Bonds: 909
        Info 1)    Backbone bonds: Protein: 231  DNA: 0
        Info 1)    Residues: 58
        Info 1)    Waters: 0
        Info 1)    Segments: 1
        Info 1)    Fragments: 1   Protein: 1   Nucleic: 0
\end{verbatim}

There are several types of fragments.  Protein and nucleic
fragments are homogeneous; either all proteins, or all nucleic acids.
However, it is possible for a protein to be connected to a
nucleic acid or some other non-protein.  
When this occurs, a warning message is printed, as in:
\index{bonds!unusual}
\begin{verbatim}
        Warning 1) Unusual bond between residues 1 and 2
\end{verbatim}
These warnings will occur with terminal amino acids, zinc
fingers, myristolated residues, and poorly defined structures.

\section{Babel interface}
\label{ug:topic:filetypes:babel}
\VMD\ can use the program \index{Babel}Babel, if installed, to
translate a wide variety of different molecular data files into the
PDB format.  Not all of these have been tested for use with \VMD, 
so your results may vary.
\VMD\ only uses Babel to read files and does not
allow the use of Babel to save files to other formats.
\index{environment variables!VMDBABELBIN}
The {\tt VMDBABELBIN} 
\hyperref{environment variable}{environment variable [\S }{]}{ug:exec_env:variables}
is used to specify the absolute path to the the Babel executable 
(including the executable name).
For more information about Babel, see \htmladdnormallink{{\tt
http://smog.com/chem/babel/}}{http://smog.com/chem/babel/}.  
VMD currently supports version 1.6 of Babel.  

\section{Raster3D file format}

In addition to the molecular file formats, \VMD\ can read the input
file for \index{Raster3D}Raster3D.  (Raster3D converts an input file
into a shaded raster image for use in making high quality pictures.
It is often used with MolScript.)  The ability to read Raster3D allows
users to view MolScript files in 3D and incorporate special images
into the display without having to edit the \VMD\ code.  
The file format, which is part of the Raster3D documentation,
describes a simple collection of triangles, spheres, and cylinders
with either flat or spherical ends.  Each shape is colored by an RGB
triplet.

Certain newer Raster3D objects are ignored, such as quadrics. Also,
nearly all of the header information is ignored---most notably, the
viewing matrix.  Raster3D uses many cylinders with spherical
(rounded) ends. \VMD\ deliberately omits these rounded ends since the
resultant image would be very slow to render interactively.
\VMD\ uses a fixed size palette of colors, each triplet is
converted into its ``nearest'' indexed color.  This may cause images
to be colored slightly differently than expected.

